\documentclass[11pt]{scrartcl}
\usepackage{otis}

\begin{document}

\title{Quality requirements for OTIS contributions}
\maketitle

\section*{Why are standards so high?}
OTIS prides itself on extremely high standards for solutions on production.
As the number of students has increased, so have the standards;
e.g.\ a lot of old solutions written by Evan are missing or incomplete,
or otherwise would not meet modern OTIS requirements.

Dealing with these old crappy solutions has been a major sore point,
and it has been a long and slow process to slowly improve them.
Every badly written solution eventually leads to some student asking Evan
what the heck is going on, and Evan having to rush to fix the solution.
We are still paying off this ``quality debt'', years and years later.

To prevent the further buildup of quality debt, we therefore are asking that
\alert{all future contributions to OTIS are production quality}.
That means a submission needs to be perfect before
it will get added into the OTIS system,
because we don't want our to-do list of things to clean up to get any longer.

Without further ado, here are the requirements:

\tableofcontents

\newpage

\section{Solutions must be complete and your own writing}
\alert{You MUST NOT copy the wording of someone else's solution},
at least not without permission.
However, it's okay if the ideas in the solution are not your own,
or someone else found an isomorphic solution to yours.

You \alert{must write up a complete solution} to each problem.

\alert{Links to AoPS are not acceptable as solutions},
even links to posts written by yourself.
You need to actually format the solution in \LaTeX\ properly and submit it.

\section{Write out a full $7$-point solution}
You must \alert{write out all details, like in a competition setting}.
Do not include phrases like ``exercise to the reader'',
``can be checked that'', etc.

\section{Follow the \LaTeX\ style guide to a tee}
Follow Evan's \LaTeX\ style guide (yes, I know it's long):
\begin{center}
  \Large
  \url{https://web.evanchen.cc/latex-style-guide.html}.
\end{center}
In addition, make sure that you don't have any obvious syntax issues, for example:
\begin{itemize}
\ii Capitalize the first word of each sentence.
Also, avoid starting any sentence in math mode.

\ii Make sure you have a space after full stops (periods) and commas, but not before.

\ii Use notation consistent with OTIS.
\end{itemize}

\section{Isolate claims and include paragraph breaks}
\alert{Include paragraph breaks when appropriate}.
Don't submit a solution that's a single multi-page paragraph.

Separate claims/sub-lemmas using the \texttt{claim*} environment
(from \texttt{evan.sty}), and their proofs with \texttt{proof}.
For example, the code
\begin{lstlisting}[language=TeX]
\begin{claim*}
  The function $f$ is injective.
\end{claim*}
\begin{proof}
  If $f(a) = f(b)$, then consider $P(a,b)$ and do magic.
\end{proof}
\end{lstlisting}
will produce
\begin{claim*}
  The function $f$ is injective.
\end{claim*}
\begin{proof}
  If $f(a) = f(b)$, then consider $P(a,b)$ and do magic.
\end{proof}

\section{Include figures}
You must \alert{include diagrams whenever relevant}
for all solutions. This includes not only geometry problems,
and any combinatorics problem for which a diagram makes sense ---
for example, grid combinatorics should be illustrated,
graph theory solutions should have graphs in them, etc.

\alert{Diagrams should be drawn in Asymptote}.
If you don't know how to use Asymptote,
you'll either have to learn or ask someone to help draw for you.

As a specific example, consider the USAMO 2022 solutions file:
\begin{quote}
  \url{https://web.evanchen.cc/exams/USAMO-2022-notes.pdf}
\end{quote}
Notice that
\begin{itemize}
  \ii USAMO 2022/1 is a grid combo problem, and the grid and operation are
  actually drawn.
  \ii USAMO 2022/2 is a quasi-geometry problem, and of course has a diagram.
  \ii USAMO 2022/3 is a functional equation, and a diagram doesn't make sense here.
  \ii USAMO 2022/4 is a Diophantine equation, and a diagram doesn't make sense here.
  \ii USAMO 2022/5 doesn't have an Asymptote diagram,
  but it instead has a table showing an example of a construction,
  complete with careful alignment and red boldface showing a well-chosen example.
  \ii USAMO 2022/6 is a graph theory problem and chock-full of figures
  to show the extremely complicated algorithm.
\end{itemize}

As a general rule, if you needed to draw a picture when solving the problem,
you should have diagrams in the solution.
If you defined a procedure or coloring or labeling or whatever,
and it would help to illustrate an example, you should do so.

The code for the diagram can use boilerplate from \texttt{evan.sty},
but it should not depend on the nonstandard \texttt{olympiad.asy} or \texttt{cse5.asy}.
In general, the diagram should be surrounded by center and asy, that is:
\begin{lstlisting}[language=TeX]
\begin{center}
\begin{asy}
  draw(unitcircle);
\end{asy}
\end{center}
\end{lstlisting}

\end{document}
